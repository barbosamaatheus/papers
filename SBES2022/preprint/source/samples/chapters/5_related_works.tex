\section{Trabalhos relacionados}
Nesta seção descrevemos alguns dos estudos anteriores que usamos como base de evidência para nosso estudo e trabalhos relacionados.

Vários pesquisadores já investigaram sobre os conflitos de \emph{merge} e como eles afetam os desenvolvedores e a produtividade de forma geral \cite{10.1145/383876.383878, 4228648, Sarma, Bird2012AssessingTV, 8094445, 6915251, 9625780, 10.5555/1781794.1781836}. Em contraste, aqui focamos na detecção de conflitos semânticos, que normalmente são mais difíceis de se detectar e resolver. Não conhecemos trabalhos que analisam o impacto de conflitos semânticos em produtividade. No entanto, ferramentas como a que discutimos aqui são essenciais para ajudar a entender o impacto de conflitos semânticos em produtividade.

%Pesquisadores também analisaram outras maneiras de detectar ou prevenir conflitos, minimizando assim o impacto na produtividade. Sarma et al. \cite{Sarma} apresenta uma ferramenta projetada para reduzir conflitos, notificando os desenvolvedores sobre mudanças paralelas no mesmo artefato. Com esse intuito, existem também outras ferramentas.  \cite{10.1145/1810295.1810339, DBLP:journals/corr/abs-1105-0768, 10.5555/2486788.2486884, 10.1145/2675133.2675177,10.1145/2950290.2950339,10.1007/s10515-017-0227-0, Klissiomara, 10.1145/3238147.3241983, 6475431, 1000449}.

Com o intuito de progredir no processo de detecção de conflitos de \emph{merge} de forma mais precisa e reduzir esforços da integração, foram criadas diversas ferramentas. Westfechtel \cite{10.1145/111062.111071} e Buffenbarger \cite{Buffenbarger} foram pioneiros em propor soluções para realizar \emph{merge} usando estruturas de arquivos. Posteriormente, outros pesquisadores implementaram soluções baseadas em construções específicas de linguagens de programação, como \emph{Java} \cite{10.1007/s10515-006-0002-0} e \emph{C++} \cite{Cdiff}.
Existem também várias ferramentas de \emph{merge} avançadas que usam a estrutura sintática dos programas integrados \cite{10.1145/2025113.2025141, 7321191, 10.1145/3133883, 8952450, 6494912, 10.1109/ICSE-Companion.2019.00117, 10.1109/ASE.2019.00097, 10.1145/3474624.3474646}, mas nenhuma delas captura conflitos semânticos dinâmicos como os ilustrados aqui.

Para de fato detectar conflitos semânticos, Da Silva \emph{et al} \cite{LeusonSilva2020} propuseram uma abordagem baseada em geração de testes unitários automatizados como especificações parciais para detecção de interferência em cenários de integração. Com 38 cenários testados, a abordagem conseguiu detectar 4 casos com conflitos (verdadeiros positivos), 11 casos falsos negativos (28,95\%) e nenhum falso positivo. Já em nosso trabalho, utilizamos uma abordagem diferente, baseada em análise estática. Os resultados para os 78 cenários da amostra utilizada apontam um percentual maior de falsos positivos (7,8\% para \emph{intraprocedural} e 7,4\% para \emph{interprocedural}) e falsos negativos (34,4\% para \emph{intraprocedural}). A nossa abordagem foi superior na comparação de falsos negativos \emph{interprocedural} (27,9\%). Contudo, argumentamos que a análise proposta neste trabalho é somente uma parte de uma solução maior utilizando analises estáticas, enquanto o trabalho de Da Silva é uma solução completa. A amostra de 38 cenários utilizada no trabalho de testes é um subconjunto da amostra utilizada em nosso trabalho. Nesse sentido, analisamos os cenários detectados por ambas as ferramentas e constatamos serem cenários diferentes. Dessa forma, entendemos que uma combinação das duas abordagens poderia ser usada em trabalhos futuros, buscando aumentar a quantidade de cenários detectados e reduzir o número de erros, principalmente os falsos negativos. Além disso, existem outros trabalhos importantes que seguem a linha de ferramentas baseadas em testes \cite{brun, 10.1145/2786805.2803208, 10.1145/2568225.2568300}, mas estes são baseados em testes do projeto, que muitas vezes não são suficientes para detectar conflitos.

Por fim, ferramentas baseadas em estratégias de análise estática também foram propostas \cite{10.1145/201055.201056, 10.1145/65979.65980, 336770, 10.1145/131736.131756, Horwitz1989IntegratingNV, 10.1145/3276535, 1113835899}. Barros Filho \cite{InformationFlowRoberto} também propôs a utilização de análises com o objetivo principal de entender se o Information Flow Control (IFC) pode ser utilizada para indicar a presença de conflitos semânticos dinâmicos entre as contribuições dos desenvolvedores em cenários de \empty{merge}. Essa análise se mostrou capaz de detectar casos de interferência, porém com uma taxa de 57.14\% de falsos positivos. Contudo, a análise implementada utilizava um grafo complexo para representar as dependências entre unidades do código, o que faz com que a análise consuma muitos recursos computacionais. Em nosso trabalho, apresentamos uma análise mais simplificada que busca detectar apenas conflitos de OA. Avaliamos a análise proposta utilizando uma amostra maior do que o dobro da amostra de Barros Filho e obtivemos um número de falsos positivos cerca de 87\% menor.